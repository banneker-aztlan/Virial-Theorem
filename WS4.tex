% 
\documentclass[11pt]{article}
\usepackage[left=2cm,right=2cm,top=2cm,bottom=2cm]{geometry} 
\geometry{letterpaper}
\usepackage{amssymb, mathtools}
\usepackage{amsmath, ulem}
\usepackage{xcolor}
\usepackage{mathabx}
\usepackage{hyperref}

% = Customize Section Headings =
\usepackage[noindentafter]{titlesec}    % default to noindent after
% sections
\usepackage{graphicx, enumitem}
\usepackage{hyperref}
\titleformat{\subsection}[runin]{\bf}{\small}{\compact}{}[]
\titleformat{\section}{\bf}{\large}{\compact}{}[]
\titlespacing{\section}{0pt}{20pt}{2pt}
\titlespacing{\subsection}{1pt}{8pt}{5pt}

% = Common Astrophysics Symbols =
\newcommand{\rsol}{R_\odot}
\newcommand{\msol}{M_\odot}
\newcommand{\lsol}{L_\odot}
\newcommand{\E}[1]{\hspace{-0.01in}\times\hspace{-0.02in}10^{#1}}  % Scientific notation, '3\E{8}' --> '3x10^8'
\newcommand{\sol}[1]{{\noindent\color{blue} #1}}     % Solutions in blue
\newcommand{\power}[2]{\ensuremath{#1\times 10^{#2}}}

\newcommand{\note}[1]{{\noindent\color{red} \textbf{#1}}} % Notes in red
\newcommand{\trf}[1]{\textrm{\footnotesize{#1}}}
\newcommand{\trt}[1]{\textrm{\tiny{#1}}}
\newcommand{\then}{ \hspace{0.1in} \Longrightarrow \hspace{0.1in} }

\newcommand\plotoneman[2]{\centering \leavevmode
  \includegraphics[width=#2\linewidth]{#1}}

\renewcommand{\section}[1]{\textbf{\underline{#1}}}

% = Setup Header =
\usepackage{fancyhdr}
\pagestyle{fancy}
\renewcommand{\headrulewidth}{0pt}   % remove fancyhdr header line
\lhead{Dynamics \& Virial Theorem}
\rhead{Banneker Institute 2017}
\cfoot{\thepage}

\title{Worksheet 4\\Return of the Virial Theorem\\Instructor: Ben Cook}
\date{}

\begin{document}
\maketitle
\thispagestyle{fancy}                   % add header to first page

\vspace{-0.6in}

Last time, we derived the complete form of the Virial Theorem:
\begin{align}
  \frac{1}{2} \frac{d^2I}{dt^2} &= 2K + \sum_i F_i
  r_i\label{eq.Virial.1}
  \intertext{We also argued (but did not prove) that the third term
    represents the potential energy $U$ of the system, giving us:}
  \frac{1}{2} \frac{d^2I}{dt^2} &= 2K - (n+1) U\label{eq.Virial.2}
  \intertext{Finally, we showed that the time-average of the left-hand
  term goes to zero if the system is in equilibrium (and the
  time-average is over an appropriately long time period), giving us:}
  2\left<K\right> &= (n+1) \left<U\right> \label{eq.Virial.3}
  \intertext{For the familiar case of a gravitational system ($n=-2$),
    we recover:}
  2\left<K\right> &= -\left<U\right> \label{eq.Virial.4}
\end{align}

In these final exercises, we will play with these forms of the
Virial Theorem in a variety of scenarios. 

\begin{enumerate}
\item \section{Hook's Law}
  
  Consider the classic ``mass on a spring'' problem (1D, no gravity or
  friction). A mass $m$ is attached to a horizontal spring that has
  spring constant $k$ and is in equilibrium at position $x=0$.

  \begin{figure*}[hbt]
    \plotoneman{spring}{0.3}
    \caption{Figure for problem 1.}
  \end{figure*}

  Hook's law tells us the force on the mass when it is at position
  $x$ is:
  \begin{align*}
    F&= -kx\, \\
    \intertext{or alternatively,}
    m \frac{d^2 x}{dt^2} &= - kx\\
    \intertext{One solution to this equation of motion is sinusoidal
      motion:}
    x(t) &= A \cos(\omega t)\,,\\
    \intertext{where the frequency is given by $\omega = \pm \sqrt{k /
      m}$.}
  \end{align*}

  \begin{enumerate}
  \item
    Solve for the kinetic energy $K(t)$ of the mass as a function of time.
  \item
    Solve for $U(x)$ as a function of position, using $x=0$ as the
    potential energy reference point.
  \item
    Solve for the potential energy $U(t)$ as a function of time.
  \item
    Solve for the total energy $E(t)$. Does this energy change in
    time?
  \item
    Confirm that equation \ref{eq.Virial.3} is correct for this system
    ($n=+1$) by taking the time average of $K$ and $U$ over a period
    $P = 2\pi\omega$. For your reference, the needed time-averages are:
    \begin{align*}
      \left<\cos^2(\omega t)\right>_{2\pi\omega} &=
      \left<\sin^2(\omega t)\right>_{2\pi\omega} = \frac{1}{2}
    \end{align*}
  \item
    Compute the moment-of-inertia $I(t) = m x^2(t)$ as a function of
    time.
  \item
    Confirm that the non-averaged Virial Theorem (eq.~\ref{eq.Virial.2})
    is correct at \textit{all times} for this system.
  \end{enumerate}

\item
  \section{Dynamical Timescale: How Long is Long Enough?}

  We've talked about applying eq.~\ref{eq.Virial.3} whenever
  time-averages are taken over ``long enough'' timescales where a
  system can be said to be in equilibrium. Here, we'll derive a good
  rule-of-thumb for determining whether a system is in equilibrium.

  Imagine that the kinetic energy of a gravitational system (star,
  cluster, etc.) suddenly disappeared ($K = 0$). How long would it
  take before the system collapses entirely?

  Equation \ref{eq.Virial.2} tells us that, in this case:
  \begin{align*}
    \frac{1}{2} \frac{d^2I}{dt^2} &= U\,.\\ \intertext{Let's now make
      an order-of-magnitude estimate (dropping the factor of
      $\frac{1}{2}$) for this process takes. If the system experiences
      a change in moment-of-inertia $\Delta I$ in time $\Delta t$, we estimate:}
    \frac{d^2I}{dt^2} &\approx \frac{\Delta I}{(\Delta t)^2} = U\,.
    \intertext{If the system collapses entirely, then the
      moment-of-inertia will go to zero, and $\Delta I = -I$. How much
      time will this take?}  \frac{-I}{(\Delta t)^2} &\approx
    U\\ (\Delta t)^2 &\approx
    \frac{-I}{U}\\ \intertext{Substituting in the moment of
      inertia $I$ and potential energy $U$ of a constant-density
      sphere:}
    (\Delta t)^2 &\approx \frac{-\frac{2}{5} M
        R^2}{-\frac{3}{5}\frac{GM^2}{R}}\\
    \Aboxed{\tau_d &\equiv
      \sqrt{\frac{2 R^3}{3 G M}}}
  \end{align*}
  
  This is the definition of the \textit{dynamical time} $\tau_d$, the
  shortest timescale over which the moment of inertia could change
  significantly. \uline{As long as the system does not change drastically
  (any $\Delta I \ll I$) on the timescale of $\tau_d$, then it can be
  considered in dynamical equilibrium, and it is valid to use
  eq.~\ref{eq.Virial.3}.}
  \begin{enumerate}
  \item
    Compute $\tau_d$ for the following objects:
    \begin{itemize}
    \item
      The Sun (use the values on the final page of
      the worksheet).
    \item
      The Milky Way ($M\approx 10^{12} M_\odot$, $R\approx 100$ kpc).
    \item
      The Coma Cluster ($M\approx 10^{15} M_\odot$, $R\approx 1$ Mpc).
    \item
      A theoretical ``Super-Duper-Cluster'' with mass $M\approx 10^{16}
      M_\odot$ and radius $R = 30$ Mpc.
    \end{itemize}
  \item
    Which of these objects do you think are likely to be in equilibrium?
    Think about whether you would expect these systems to change
    drastically within the dynamical timescales calculated above.  How
    might you judge the ``stability'' of each system?  Rank them in order
    of decreasing ``stability'' according to this judgment.
  \end{enumerate}

\item
  \section{Ideal Gas Law}
  
  Consider a cubic box (side length $s$) with $N$ gas particles at
  temperature $T$ and average pressure $P$, with the center of the box
  at the origin.

  Because the box is in equilibrium, we will use the Virial Theorem of
  the form:
  \begin{align}
    2\left<K\right> &= - \left<\sum_i F_i r_i\right> \label{eq.Virial.5}
  \end{align}
  \begin{enumerate}
  \item
    What is $2\left<K\right>$? (\textit{HINT: think thermal energy})
  \item
    What is the average \textit{TOTAL} force $\sum_i F_i$ applied by
    the gas particles on all surfaces of the box? By Newton's third
    law, the gas particles will feel a total force with the same
    magnitude, but in the opposite direction.
  \item
    At what position $r$ from the center of the box is this force
    \textit{always} applied to the particles?
  \item
    Use your answers to the above, along with eq.~\ref{eq.Virial.5} to
    derive the Ideal Gas Law:
    \begin{align*}
      P V &= N k T\,.
    \end{align*}
    If you're getting stuck with a minus sign, remember to consider
    what direction the force $\sum_i F_i$ on the gas particles is in,
    relative to their displacement $r$ from the center of the box.
  \end{enumerate}

\item
  \section{Radius of Electron Orbitals}

  We will now use the Virial Theorem to derive the radius $r_n$ of the
  n$^{\mathrm{th}}$ Hydrogen orbital. We are playing fast-and-loose
  with Quantum Mechanics (the ``time-averages'' $\left<\cdots\right>$
  are really ``quantum expectation values'' $\left<\cdots\right>$) but
  our answers are accurate to factors of a few.

  Note: this problem is done in CGS units. 

  \begin{enumerate}
  \item
    The electron is bound to the Hydrogen nucleus via an $r^{-2}$ central
    force. The potential energy of the electron at radius $r$ is given
    by:
    \begin{align}
      U(r) &= -\frac{e^2}{r}\,. \label{eq.electron}
    \end{align}
    Use the Virial Theorem to derive the expectation value of total
    energy $\left<E(r)\right>$ of an electron as a function of
    position.
  \item
    From quantum mechanics, we can derive that an electron in the
    n$^{\textrm{th}}$ eigenstate has energy:
    \begin{align}
      E_n &= -\frac{1}{2}\frac{m_e e^4}{n^2 \hbar^2}\,.\label{eq.quantum}
    \end{align}
    To an order of magnitude, we can say:
    \begin{align*}
      \left<\frac{1}{r_n}\right> &\approx
      \frac{1}{\left<r_n\right>}\,.
    \end{align*}
    Using the above, prove that:
    \begin{align*}
      \left<r_n\right> &\approx n^2 a_0\,,\\
      \intertext{where $a_0$ is the ``Bohr Radius'':}
      a_0 &\equiv \frac{\hbar^2}{m_e e^2}\,.
    \end{align*}
    
  \end{enumerate}
  
\end{enumerate}

\section{Useful Constants and Conversions}\\

\begin{tabular}{|l|l|l|l|}
  \hline
  Name & Symbol & Value (SI) & Value (CGS)\\\hline\hline
  %=======================
  Mass of Sun & $M_\odot$ & \power{2.0}{30} kg & \power{2.0}{33}
  g\\\hline
  Radius of Sun & $R_\odot$ & \power{7.0}{8} m & \power{7.0}{10} cm\\\hline
  Luminosity of Sun & $L_\odot$ & \power{3.9}{26} J s$^{-1}$ &
  \power{3.9}{33} erg s$^{-1}$\\\hline
  Gravitational Constant & $G$ & \power{6.7}{-11} m$^3$kg$^{-1}$s$^{-2}$ &
  \power{6.7}{-8} cm$^3$g$^{-1}$s$^{-2}$\\\hline
  Kilometer & km & $10^3$ m & $10^5$ cm \\\hline
  Kiloparsec & kpc & \power{3.1}{19} m & \power{3.1}{21} cm\\\hline
  Megaparsec & Mpc & \power{3.1}{22} m & \power{3.1}{24} cm\\\hline
  Year & yr & \power{3.1}{7} s & -- \\\hline
  Giga-year & Gyr & \power{3.1}{16} s & -- \\\hline
\end{tabular}


\end{document}
