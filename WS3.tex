% 
\documentclass[11pt]{article}
\usepackage[left=2cm,right=2cm,top=2cm,bottom=2cm]{geometry} 
\geometry{letterpaper}
\usepackage{amssymb, mathtools}
\usepackage{amsmath, ulem}
\usepackage{xcolor}
\usepackage{mathabx}
\usepackage{hyperref}

% = Customize Section Headings =
\usepackage[noindentafter]{titlesec}    % default to noindent after
% sections
\usepackage{graphicx, enumitem}
\usepackage{hyperref}
\titleformat{\subsection}[runin]{\bf}{\small}{\compact}{}[]
\titleformat{\section}{\bf}{\large}{\compact}{}[]
\titlespacing{\section}{0pt}{20pt}{2pt}
\titlespacing{\subsection}{1pt}{8pt}{5pt}

% = Common Astrophysics Symbols =
\newcommand{\rsol}{R_\odot}
\newcommand{\msol}{M_\odot}
\newcommand{\lsol}{L_\odot}
\newcommand{\E}[1]{\hspace{-0.01in}\times\hspace{-0.02in}10^{#1}}  % Scientific notation, '3\E{8}' --> '3x10^8'
\newcommand{\sol}[1]{{\noindent\color{blue} #1}}     % Solutions in blue
\newcommand{\power}[2]{\ensuremath{#1\times 10^{#2}}}

\newcommand{\note}[1]{{\noindent\color{red} \textbf{#1}}} % Notes in red
\newcommand{\trf}[1]{\textrm{\footnotesize{#1}}}
\newcommand{\trt}[1]{\textrm{\tiny{#1}}}
\newcommand{\then}{ \hspace{0.1in} \Longrightarrow \hspace{0.1in} }

\newcommand\plotoneman[2]{\centering \leavevmode
  \includegraphics[width=#2\linewidth]{#1}}

\renewcommand{\section}[1]{\textbf{\underline{#1}}}

% = Setup Header =
\usepackage{fancyhdr}
\pagestyle{fancy}
\renewcommand{\headrulewidth}{0pt}   % remove fancyhdr header line
\lhead{Dynamics \& Virial Theorem}
\rhead{Banneker Institute 2017}
\cfoot{\thepage}

\title{Worksheet 3\\The Virial Theorem Strikes Back\\Instructor: Ben Cook}
\date{}

\begin{document}
\maketitle
\thispagestyle{fancy}                   % add header to first page

\vspace{-0.6in}

Last time, we derived the complete form of the Virial Theorem:
\begin{align}
  \frac{1}{2} \frac{d^2I}{dt^2} &= 2K + \sum_i F_i
  r_i\label{eq.Virial.1}
  \intertext{We also argued (but did not prove) that the third term
    represents the potential energy $U$ of the system, giving us:}
  \frac{1}{2} \frac{d^2I}{dt^2} &= 2K - (n+1) U\label{eq.Virial.2}
  \intertext{Finally, we showed that the time-average of the left-hand
  term goes to zero if the system is in equilibrium (and the
  time-average is over an appropriately long time period), giving us:}
  2\left<K\right> &= (n+1) \left<U\right> \label{eq.Virial.3}
  \intertext{For the familiar case of a gravitational system ($n=-2$),
    we recover:}
  2\left<K\right> &= -\left<U\right> \label{eq.Virial.4}
\end{align}

In these final exercises, we will play with these forms of the
Virial Theorem in a variety of scenarios. \\

%\section{Proving the Virial Theorem}\\
%
%The time has finally come to \textit{prove} the Virial Theorem! Don't
%worry: while this may seem like a daunting task, it's actually quite
%straight-forward if you take it a step at a time.
%
%\begin{enumerate}
%\item
%  We will begin with the definition of the \textit{moment of inertia} $I$
%  for a large system of particles, which should look
%  somewhat familiar. The moment of inertia simply depends on the mass
%  $m$ and position $r$ of each particle in the system. In both the
%  vector and scalar versions:
%  \begin{align*}
%    I &\equiv \sum_i m_i \mathbf{r}_i \cdot \mathbf{r}_i\quad (\mathrm{vector})\\
%     &\equiv \sum_i m_i r_i^2 \quad (\textrm{1-D})\,.
%  \end{align*}
%  Feel free to continue with the proof either in 1-D, or to work with
%  the vector forms, if you're comfortable with vector dot-products.
%  
%  \begin{enumerate}
%  \item
%    Take one derivative w.r.t.~time to show that:
%    \begin{align*}
%      \frac{dI}{dt} &= 2Q\,,\\
%      \intertext{where $Q$ is the so-called ``Virial of Claussius'',
%        given by:}
%      Q &= \sum_i \mathbf{p}_i \cdot \mathbf{r}_i \quad (\mathrm{vector})\\
%       &= \sum_i p_i r_i \quad (\textrm{1-D})
%    \end{align*}
%  \item
%    Take a second derivative w.r.t~time and show that:
%    \begin{align}
%      \Aboxed{\frac{1}{2} \frac{d^2I}{dt^2} &= 2K + \sum_i \mathbf{F}_i \cdot
%      \mathbf{r}_i} \quad (\textrm{Virial Theorem, full vector form}) \label{eq.Virial.vector} \\
%      \Aboxed{\frac{1}{2} \frac{d^2I}{dt^2} &= 2K + \sum_i F_i r_i}
%      \quad (\textrm{Virial Theorem, full 1-D form})\label{eq.Virial.scalar}
%    \end{align}
%  \end{enumerate}
%\end{enumerate}
%\section{Final Steps}
%
%If the system of particles are interacting via a \textit{power-law
%  central force} which obeys:
%\begin{align*}
%  F &\propto - r^n\,,
%\end{align*}
%for some power $n$\footnote{$n=-2$ for gravity or electrostatics, $n=1$
%  for Hook's Law for springs, etc.}, then the potential energy of
%the entire system can be proven to be:
%\begin{align}
%  U &= -\frac{1}{n+1} \sum_i \mathbf{F}_i \cdot \mathbf{r}_i\quad(\mathrm{vector})\label{eq.U.vector}\\
%  &= -\frac{1}{n+1} \sum_i F_i r_i \quad (\textrm{1-D})\label{eq.U.scalar}
%\end{align}
%
%The \textit{time-average} of the left-hand side over some time
%interval $\tau$ is:
%\begin{align}
%  \left<\frac{1}{2}\frac{d^2I}{dt^2}\right>_\tau &=
%  \frac{1}{2}\left( \frac{1}{\tau} \int_0^\tau \frac{d^2I}{dt^2}
%  dt\right) \nonumber\\ &=
%  \frac{1}{2\tau}\left(\frac{dI}{dt}\bigg|_\tau -
%  \frac{dI}{dt}\bigg|_0\right) \nonumber\\ \intertext{If the
%    system is in equilibrium, then the term in parentheses is
%    bounded (will never reach infinity). Then, by averaging over a
%    large enough $\tau$:}
%  \Aboxed{\left<\frac{1}{2}\frac{d^2I}{dt^2}\right>_\tau&\approx 0}
%  \quad(\textrm{If $\tau$ large enough})\label{eq.time}
%\end{align}
%
%If the system is periodic, then we can choose $\tau$ so that the
%time-averages $\left<\cdots\right>$ are over a period, so that the
%difference in parentheses goes to zero. If it is not strictly
%periodic, then you can just choose an appropriately large value of
%$\tau$. If the system is in ``equilibrium'', then the terms in the
%parentheses will remain small, and
%$\left<\frac{1}{2}\frac{d^2I}{dt^2}\right>$ will approach zero.
%
%\begin{enumerate}[resume]
%\item
%  Substitute equations \ref{eq.U.vector} or \ref{eq.U.scalar} into
%  the complete Virial Theorem (\ref{eq.Virial.vector} or
%  \ref{eq.Virial.scalar}), and then take a time-average of both
%  sides. Use the limit given in \ref{eq.time} to conclude:
%  \begin{align}
%    \Aboxed{2\left<K\right> &= (n+1) \left<U\right>}\\ (\textrm{Virial
%      Theorem, system in}&\textrm{ equilibrium, central force
%      $F\propto-r^n$})\nonumber
%  \end{align}
%\end{enumerate}
%
\section{Dark Matter in Galaxy Clusters}\\

The most well-known and frequently-cited use of the Virial Theorem was
in the discovery of dark matter. By applying the theorem to the
velocities of galaxies in a galaxy cluster, we can make an estimate of
the total mass of the cluster.

\begin{enumerate}
\item
  We will model a galaxy cluster as a sphere with mass $M$ (comprised
  of many galaxies) and radius $R$.
  \begin{enumerate}
  \item
    We can use the redshifts of each galaxy to try to measure the
    velocity-dispersion $\left<v^2\right>$ of the cluster. But we can
    only actually measure the component $\left<v_r^2\right>$
    \textit{along the line of sight}; we can't hope to measure their
    velocity in the plane of the sky (we'd have to wait for millions
    of years).

    How does the true 3D \textit{velocity dispersion}
    $\left<v^2\right>_{3D}$ relate to the observed radial velocity
    dispersion $\left<v_r^2\right>$, if we assume the clusters are on
    randomly-oriented orbits?
    
  \item
    Using the Virial Theorem to connect the kinetic and potential
    energies of the cluster, derive an estimate of the total mass $M$
    as a function of the velocity dispersion $\left<v^2\right>_{3D}$
    and the radius $R$.

  \item
    You observe the galaxies in the cluster to have a radial velocity
    dispersion $\left<v_r^2\right> = 10^6$ km$^2$s$^{-2}$, and estimate
    it to have a size $R = 1.5$ Mpc. What is your prediction for the
    total mass $M$ in solar masses ($M_\odot$)?

  \item
    Measuring the light coming from all the cluster galaxies, you
    measure the total luminosity of the cluster to be $L = 10^{13}
    L_\odot$. What is the ``mass-to-light'' ratio $M/L$ in solar
    units. Does anything seem strange?

  \item
    The Virial Theorem implies there is much more mass in clusters
    than you can explain from the observed
    galaxies. \underline{Congratulations, you've just discovered Dark
      Matter!}

    But, being the skeptical scientist you are, you want to make sure
    you've considered all other possibilities. What are some
    alternative ways you could explain your results?

    \begin{itemize}
    \item
      Think about the assumptions you've made in this analysis. What
      ways can you think of to argue that they are valid?
    \item
      Maybe there were errors in the data you used. How far off would
      different measurements have to be to explain your results? Think critically
      about how each value was \textit{actually measured} by
      astronomers, and where the most likely places are to have gotten
      things wrong.
    \end{itemize}
  \end{enumerate}
\end{enumerate}  

\section{Fritz Zwicky, Vera Rubin, and Dark Matter}\\

The analysis you just completed was first published by Caltech
Astronomer Fritz Zwicky in 1933, not long after it was first
conclusively proven that the ``nebulae'' seen around the Milky Way
were in fact distant galaxies. Zwicky compiled measurements of
galactic redshifts in several clusters, most notably the Coma cluster,
and used the Virial Theorem to argue that there must be huge amounts
of ``dark matter'' there to account for the large masses.

\begin{enumerate}[resume]
\item
  Read the provided pages from Zwicky's 1933 paper (translated from
  German) which first argued for the existence of dark matter. It
  should be quite straight-forward, as you've just worked through the
  same exact problem!

  If you're interested, you can find the full article at

  \href{http://spiff.rit.edu/classes/phys440/lectures/gal\_clus/zwicky\_1933\_en.pdf}{http://spiff.rit.edu/classes/phys440/lectures/gal\_clus/zwicky\_1933\_en.pdf}.
\end{enumerate}

Zwicky was an edgy, often confrontational personality, whose life now
makes for a very entertaining study. But at the time, Zwicky's
discovery of dark matter was largely ignored, as were many of his
other groundbreaking ideas. In a short biography of Zwicky, Professor
Stephen Maurer wrote: ``When researchers talk about neutron stars,
dark matter, and gravitational lenses, they all start the same way:
'Zwicky noticed this problem in the 1930s. Back then, nobody
listened...'''.

Zwicky's dark matter hypothesis finally gained acceptance in the
1970s, with huge credit due to Vera Cooper Rubin and Kent Ford of the
Carnegie Institute. Rubin and Ford measured the rotation curve of the
Andromeda Galaxy and showed that dark matter must exist around
galaxies (a problem you also have tackled, back in Worksheet 1). Rubin
was a true vanguard for women in astronomy (only the second woman
elected to the National Academy of Sciences), and overcame large
obstacles in a field determined to deter her from success. Sadly, she
passed away last December (2016). Many in the field believed her
work strongly deserving of the Nobel Prize, but since they are not
awarded posthumously, this recognition will never be granted to her.

\begin{enumerate}[resume]
\item
  Spend some time researching the story of Fritz Zwicky's entertaining
  personality, and Vera Rubin's pioneering journey.
\end{enumerate}

\section{Hook's Law}

\begin{enumerate}[resume]
\item 
  Consider the classic ``mass on a spring'' problem (1D, no gravity or
  friction). A mass $m$ is attached to a horizontal spring that has
  spring constant $k$ and is in equilibrium at position $x=0$.

  \begin{figure*}[hbt]
    \plotoneman{spring}{0.3}
    \caption{Figure for problem 4..}
  \end{figure*}

  Hook's law tells us the force on the mass when it is at position
  $x$ is:
  \begin{align}
    F&= -kx\, \label{eq.Hook1}\\
    \intertext{or alternatively,}
    m \frac{d^2 x}{dt^2} &= - kx \label{eq.Hook2}\\
    \intertext{One solution to this equation of motion is sinusoidal
      motion:}
    x(t) &= A \cos(\omega t)\,, \label{eq.Hook3}\\
    \intertext{where the frequency is given by $\omega = \pm \sqrt{k /
      m}$.}
  \end{align}

  \begin{enumerate}
  \item
    Solve for the kinetic energy $K(t)$ of the mass as a function of time.
  \item
    Solve for $U(x)$ as a function of position, using $x=0$ as the
    potential energy reference point.
  \item
    Solve for the potential energy $U(t)$ as a function of time.
  \item
    Solve for the total energy $E(t)$. Does this energy change in
    time?
  \item
    Confirm that equation \ref{eq.Virial.3} is correct for this system
    ($n=+1$) by taking the time average of $K$ and $U$ over a period
    $P = 2\pi\omega$. For your reference, the needed time-averages are:
    \begin{align}
      \left<\cos^2(\omega t)\right>_{2\pi\omega} &=
      \left<\sin^2(\omega t)\right>_{2\pi\omega} = \frac{1}{2} \label{eq.Hook4}
    \end{align}
  \item
    Compute the moment-of-inertia $I(t) = m x^2(t)$ as a function of
    time.
  \item
    Confirm that the non-averaged Virial Theorem (eq.~\ref{eq.Virial.2})
    is correct at \textit{all times} for this system.
  \end{enumerate}

\item
  \section{Dynamical Timescale: How Long is Long Enough?}

  We've talked about applying eq.~\ref{eq.Virial.3} whenever
  time-averages are taken over ``long enough'' timescales where a
  system can be said to be in equilibrium. Here, we'll derive a good
  rule-of-thumb for determining whether a system is in equilibrium.

  Imagine that the kinetic energy of a gravitational system (star,
  cluster, etc.) suddenly disappeared ($K = 0$). How long would it
  take before the system collapses entirely?

  Equation \ref{eq.Virial.2} tells us that, in this case:
  \begin{align}
    \frac{1}{2} \frac{d^2I}{dt^2} &= U\,.\\ \intertext{Let's now make
      an order-of-magnitude estimate (dropping the factor of
      $\frac{1}{2}$) for this process takes. If the system experiences
      a change in moment-of-inertia $\Delta I$ in time $\Delta t$, we estimate:}
    \frac{d^2I}{dt^2} &\approx \frac{\Delta I}{(\Delta t)^2} = U\,.
    \intertext{If the system collapses entirely, then the
      moment-of-inertia will go to zero, and $\Delta I = -I$. How much
      time will this take?}  \frac{-I}{(\Delta t)^2} &\approx
    U\\ (\Delta t)^2 &\approx
    \frac{-I}{U}\\ \intertext{Substituting in the moment of
      inertia $I$ and potential energy $U$ of a constant-density
      sphere:}
    (\Delta t)^2 &\approx \frac{-\frac{2}{5} M
        R^2}{-\frac{3}{5}\frac{GM^2}{R}}\\
    \Aboxed{\tau_d &\equiv
      \sqrt{\frac{2 R^3}{3 G M}}}\label{eq.dynamical}
  \end{align}
  
  This is the definition of the \textit{dynamical time} $\tau_d$, the
  shortest timescale over which the moment of inertia could change
  significantly. \uline{As long as the system does not change drastically
  (any $\Delta I \ll I$) on the timescale of $\tau_d$, then it can be
  considered in dynamical equilibrium, and it is valid to use
  eq.~\ref{eq.Virial.3}.}
  \begin{enumerate}
  \item
    Compute $\tau_d$ for the following objects:
    \begin{itemize}
    \item
      The Sun (use the values on the final page of
      the worksheet).
    \item
      The Milky Way ($M\approx 10^{12} M_\odot$, $R\approx 100$ kpc).
    \item
      The Coma Cluster ($M\approx 10^{15} M_\odot$, $R\approx 1$ Mpc).
    \item
      A theoretical ``Super-Duper-Cluster'' with mass $M\approx 10^{16}
      M_\odot$ and radius $R = 30$ Mpc.
    \end{itemize}
  \item
    Which of these objects do you think are likely to be in equilibrium?
    Think about whether you would expect these systems to change
    drastically within the dynamical timescales calculated above.  How
    might you judge the ``stability'' of each system?  Rank them in order
    of decreasing ``stability'' according to this judgment.
  \end{enumerate}

\item
  \section{Ideal Gas Law}
  
  Consider a cubic box (side length $s$) with $N$ gas particles at
  temperature $T$ and average pressure $P$, with the center of the box
  at the origin.

  Because the box is in equilibrium, we will use the Virial Theorem of
  the form:
  \begin{align}
    2\left<K\right> &= - \left<\sum_i F_i r_i\right> \label{eq.Virial.5}
  \end{align}
  \begin{enumerate}
  \item
    What is $2\left<K\right>$? (\textit{HINT: think thermal energy})
  \item
    What is the average \textit{TOTAL} force $\sum_i F_i$ applied by
    the gas particles on all surfaces of the box? By Newton's third
    law, the gas particles will feel a total force with the same
    magnitude, but in the opposite direction.
  \item
    At what position $r$ from the center of the box is this force
    \textit{always} applied to the particles?
  \item
    Use your answers to the above, along with eq.~\ref{eq.Virial.5} to
    derive the Ideal Gas Law:
    \begin{align*}
      P V &= N k T\,.
    \end{align*}
    If you're getting stuck with a minus sign, remember to consider
    what direction the force $\sum_i F_i$ on the gas particles is in,
    relative to their displacement $r$ from the center of the box.
  \end{enumerate}

\item
  \section{Radius of Electron Orbitals}

  We will now use the Virial Theorem to derive the radius $r_n$ of the
  n$^{\mathrm{th}}$ Hydrogen orbital. We are playing fast-and-loose
  with Quantum Mechanics (the ``time-averages'' $\left<\cdots\right>$
  are really ``quantum expectation values'' $\left<\cdots\right>$) but
  our answers are accurate to factors of a few.

  Note: this problem is done in CGS units. 

  \begin{enumerate}
  \item
    The electron is bound to the Hydrogen nucleus via an $r^{-2}$ central
    force. The potential energy of the electron at radius $r$ is given
    by:
    \begin{align}
      U(r) &= -\frac{e^2}{r}\,. \label{eq.electron}
    \end{align}
    Use the Virial Theorem to derive the expectation value of total
    energy $\left<E(r)\right>$ of an electron as a function of
    position.
  \item
    From quantum mechanics, we can derive that an electron in the
    n$^{\textrm{th}}$ eigenstate has energy:
    \begin{align}
      E_n &= -\frac{1}{2}\frac{m_e e^4}{n^2 \hbar^2}\,.\label{eq.quantum}
    \end{align}
    To an order of magnitude, we can say:
    \begin{align}
      \left<\frac{1}{r_n}\right> &\approx
      \frac{1}{\left<r_n\right>}\,.
    \end{align}
    Using the above, prove that:
    \begin{align}
      \left<r_n\right> &\approx n^2 a_0\,,\\
      \intertext{where $a_0$ is the ``Bohr Radius'':}
      a_0 &\equiv \frac{\hbar^2}{m_e e^2}\,.
    \end{align}
    
  \end{enumerate}
  
\end{enumerate}

%\newpage
\section{Useful Constants and Conversions}\\

\begin{tabular}{|l|l|l|l|}
  \hline
  Name & Symbol & Value (SI) & Value (CGS)\\\hline\hline
  %=======================
  Mass of Sun & $M_\odot$ & \power{2.0}{30} kg & \power{2.0}{33} g\\\hline
  Radius of Sun & $R_\odot$ & \power{7.0}{8} m & \power{7.0}{10} cm\\\hline
  Luminosity of Sun & $L_\odot$ & \power{3.9}{26} J s$^{-1}$ &
  \power{3.9}{33} erg s$^{-1}$\\\hline
  Gravitational Constant & $G$ & \power{6.7}{-11} m$^3$kg$^{-1}$s$^{-2}$ &
  \power{6.7}{-8} cm$^3$g$^{-1}$s$^{-2}$\\\hline
  Kilometer & km & $10^3$ m & $10^5$ cm \\\hline
  Kiloparsec & kpc & \power{3.1}{19} m & \power{3.1}{21} cm\\\hline
  Megaparsec & Mpc & \power{3.1}{22} m & \power{3.1}{24} cm\\\hline
  Year & yr & \power{3.1}{7} s & -- \\\hline
  Giga-year & Gyr & \power{3.1}{16} s & -- \\\hline
\end{tabular}


\end{document}
