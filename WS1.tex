% 
\documentclass[11pt]{article}
\usepackage[left=2cm,right=2cm,top=2cm,bottom=2cm]{geometry} 
\geometry{letterpaper}
\usepackage{amssymb}
\usepackage{amsmath}
\usepackage{xcolor}
\usepackage{mathabx}
\usepackage{hyperref}

% = Customize Section Headings =
\usepackage[noindentafter]{titlesec}    % default to noindent after
% sections
\usepackage{graphicx, enumitem}
\usepackage{hyperref}
\titleformat{\subsection}[runin]{\bf}{\small}{\compact}{}[]
\titleformat{\section}{\bf}{\large}{\compact}{}[]
\titlespacing{\section}{0pt}{20pt}{2pt}
\titlespacing{\subsection}{1pt}{8pt}{5pt}

% = Common Astrophysics Symbols =
\newcommand{\rsol}{R_\odot}
\newcommand{\msol}{M_\odot}
\newcommand{\lsol}{L_\odot}
\newcommand{\E}[1]{\hspace{-0.01in}\times\hspace{-0.02in}10^{#1}}  % Scientific notation, '3\E{8}' --> '3x10^8'
\newcommand{\sol}[1]{{\noindent\color{blue} #1}}     % Solutions in blue
\newcommand{\power}[2]{\ensuremath{#1\times 10^{#2}}}

\newcommand{\note}[1]{{\noindent\color{red} \textbf{#1}}} % Notes in red
\newcommand{\trf}[1]{\textrm{\footnotesize{#1}}}
\newcommand{\trt}[1]{\textrm{\tiny{#1}}}
\newcommand{\then}{ \hspace{0.1in} \Longrightarrow \hspace{0.1in} }

\newcommand\plotoneman[2]{\centering \leavevmode
  \includegraphics[width=#2\linewidth]{#1}}

\renewcommand{\section}[1]{\textbf{\underline{#1}}}

% = Setup Header =
\usepackage{fancyhdr}
\pagestyle{fancy}
\renewcommand{\headrulewidth}{0pt}   % remove fancyhdr header line
\lhead{Dynamics \& Virial Theorem}
\rhead{Banneker Institute 2017}
\cfoot{\thepage}

\title{Worksheet 1\\Dynamics Overview\\Instructor: Ben Cook}
\date{}

\begin{document}
\maketitle
\thispagestyle{fancy}                   % add header to first page

\vspace{-0.6in}

During this week's Astro-Skills module, we will be discussing one of
the most useful (and simple) equations in astronomy, the
\textit{Virial Theorem}. Before we get there, however, it is important
to warm-up by reviewing a few principles from dynamics.\\

\section{Kinetic and Potential Energy}\\

For an object in motion with mass $m$ and velocity $v$, its kinetic
energy $K$ (the energy due to its motion) is given by:
\begin{equation}
  K = \frac{1}{2}mv^2\,.
\end{equation}
If multiple objects in a system (such as multiple stars in a galaxy)
are all in motion, then the total kinetic energy of the system is
simply the sum of the individual kinetic energies of each object.

Potential energy $U$ is a measure of how much stored energy an object
has (technically, the amount of work it is capable of performing) at
its current position. The difference in potential energy between two
points ($\mathbf{r}_1$ and $\mathbf{r}_2$) is determined by the force
($\mathbf{F}$, as a vector) applied along the trajectory between the points:
\begin{equation}
  \Delta U = - \int_{\mathbf{r}_1}^{\mathbf{r}_2} \mathbf{F} \cdot d\mathbf{r}\,.
\end{equation}
If working in one-dimension, then this simplifies to:
\begin{equation}
  \Delta U = -\int_{x_1}^{x_2} F dx\,.
\end{equation}
To define an object's potential energy, we must always define a
\textit{reference location}, where we define potential energy to be
zero. The choice of this location depends on the context of the
problem being approached.

Unlike with kinetic energy, the presence of multiple objects in a
system may change each object's potential energy, such as when a force
must have been applied to move two charged particles close to one
another. To calculate the total potential energy of a system, imagine
building the system up in pieces: calculate the work required to move
the first object into place, then the second, then the third, and so
on. See the final section (\textbf{Self-bound Sphere}) for an example.

The total energy $E$ of a system is simply the sum of its kinetic and
potential energies (if we neglect other complicated energies, such as
magnetic pressure, rotation, thermal energy, etc.):
\begin{equation}
  E = K + U\,.
\end{equation}


\section{Gravitation and Circular Motion}\\

Two massive objects ($M$ and $m$) separated by a distance $r$ (or
displacement vector $\mathbf{r}$) will feel a gravitational force of
attraction. If they are point-masses (or spheres) the force is simply:
\begin{align}
  \mathbf{F} &= -\frac{GMm}{r^3}\mathbf{r}\quad(\mathrm{vector})\\
  F &= -\frac{GMm}{r^2} \quad (\mathrm{scalar}).
\end{align}

An object on a circular trajectory must be subject to a
\textit{centripetal acceleration} directed towards the center of the circle
in order to remain on the trajectory. The magnitude of the
acceleration is given by the radius $r$ and velocity $v$ of the trajectory:
\begin{align}
  a_{cen} &= \frac{v^2}{r}\,.
\end{align}

\begin{enumerate}
\item
  \label{prob.orbit}
  A satellite (mass $m$) is on a circular orbit with radius $r$ from
  the center of the Earth (mass $M$).

  \begin{figure*}[hbt]
    \plotoneman{circular_orbit}{0.4}
    \caption{Figure for Problem \ref{prob.orbit}.}
  \end{figure*}

  \begin{enumerate}
  \item
    \label{prob.orbit.1}
    Solve for the satellite's velocity $v(r)$ as a function of radius,
    such that the gravitational force balances the required
    centripetal acceleration.
  \item
    \label{prob.orbit.2}
    Solve for the period $P$ of the orbit.
  \item
    \label{prob.orbit.3}
    Solve for the satellite's kinetic energy $K(r)$ as a function of
    radius.
  \item
    \label{prob.orbit.4}
    When considering gravity, it is customary to set the potential
    energy reference point at $r\rightarrow\infty$, since the force is
    infinite at the origin. Using the fact that $U(r\rightarrow\infty)
    = 0$, solve for the potential energy $U(r)$ of the satellite.

    This is important enough of a result that I'll give you the answer
    here, so you can check. The gravitational potential energy of an
    object with mass $m$ near a mass $M$ is:
    \begin{equation}
      U(r) = -\frac{GMm}{r} \label{eq.grav_pot}
    \end{equation}
  \item
    \label{prob.orbit.5}
    Compute the total energy of the satellite, $E(r)$.
  \end{enumerate}

\end{enumerate}

\section{Gauss' Law}\\

Newton proved that the gravitational force on an object outside a
uniform sphere of mass $M$ is the same as that from a point-mass with
the same mass. Even more generally, the gravitational force
at a distance $r$ from any object which is \textit{spherically
  symmetric} simply depends on the \textbf{mass enclosed within that radius},
$M(<r)$:
\begin{equation}
  F(r) = \frac{GmM(<r)}{r^2}\,.
\end{equation}

This is a consequence of ``Gauss' Law''. If you're interested in a
rigorous proof of this, talk to Ben or check Wikipedia\footnote{\href{https://en.wikipedia.org/wiki/Gauss's\_law\_for\_gravity}{https://en.wikipedia.org/wiki/Gauss's\_law\_for\_gravity}}.

\begin{enumerate}[resume]
\item
  \label{prob.Gauss}
  Consider a hollow spherical shell, with all of its mass $M$ contained
  at a radius $R$.
  
  \begin{figure*}[hbt]
    \plotoneman{shell}{0.4}
    \caption{Figure for Problem \ref{prob.Gauss}.}
  \end{figure*}

  \begin{enumerate}
  \item
    \label{prob.Gauss.1}
    What is the gravitational force $F(r)$ on a particle with mass $m$
    when \textit{outside} the sphere ($r > R$)?
  \item
    \label{prob.Gauss.2}
    What is the force $F(r)$ on it when \textit{inside} the sphere ($r
    < R$)?
  \end{enumerate}
\end{enumerate}

\section{Self-bound Sphere}

\begin{enumerate}[resume]
\item
  \label{prob.Sphere}
  In this next problem, we will compute the potential energy $U$ of a
gravitationally-bound sphere, a quantity which will arise extremely
often in Astronomy. We will make this computation by imagining
building up the sphere inside-out, and summing the potential energies
required to move each slice of the sphere into position from
infinity\footnote{This is not the only way to reach the correct
  answer. The same result can be derived outside-in, or in any
  haphazard order you want. Gauss' law requires the results to be the same.}.

Consider a sphere with mass $M$ and radius $R$ which has constant
density $\rho$ throughout the interior.

\begin{enumerate}
\item
  \label{prob.Sphere.1}
  Use dimensional analysis to make an order-of-magnitude guess for the
  potential energy of the sphere. (\textit{HINT: think about
    eq.~\ref{eq.grav_pot}, and what parameters are available in this problem.})
\item
  \label{prob.Sphere.2}
  Compute $\rho$, and use it to solve for the enclosed mass $M(<r)$
  as a function of radius. 
\item
  \label{prob.Sphere.3}
  To compute the gravitational potential energy, let's imagine that we
  have already created a small portion of the sphere with radius $r$,
  and are now adding a thin shell with thickness $\Delta r$ (see
  Figure \ref{fig.shell}).

  \begin{figure*}[hbt]
    \plotoneman{sphere}{0.3}
    \caption{Figure for Problem \ref{prob.Sphere.3}.}
    \label{fig.shell}
  \end{figure*}

  Use eq.~\ref{eq.grav_pot} to compute the additional potential energy
  $\Delta U$ (in terms of $\Delta r$) from adding this shell.

\item
  \label{prob.Sphere.4}
  Convert to the infinitesimal limit ($dU$ in terms of $dr$) and
  integrate through the whole sphere to find $U$ in terms of $M$ and
  $R$. How does this answer compare to your initial guess (\ref{prob.Sphere.1})?
\end{enumerate}

\item
  \label{prob.limits}
In \ref{prob.Sphere.1}, you used dimensional analysis to estimate the
gravitational potential for a sphere of mass $M$ and radius $R$ should
be something like $U = - \frac{GM^2}{R}$. The full solution is:
\begin{equation}
  U = - \alpha \frac{GM^2}{R}\,,
\end{equation}
where $\alpha$ is a dimensionless factor of order $1$ that is
determined by how the mass is concentrated within the sphere.

\begin{enumerate}
\item
  \label{prob.limits.1}
  What is $\alpha$ for a sphere with constant density (see
  \ref{prob.Sphere.4})?

\item
  \label{prob.limits.2}
  If the sphere had higher density in the center than in the outer
  layers, would $\alpha$ be larger or smaller than the value above?
\end{enumerate}
\end{enumerate}

\section{Milky Way Rotation Curve}

\begin{enumerate}[resume]

\item
  \label{prob.MW}
  Let's approximate the Milky Way galaxy as a sphere rather than a
  disk (this is not a terrible assumption, it turns out). We
  approximate it as having a constant density $\rho$ which extends from
  $r=0$ to some terminal radius $r=R$, at which point the density goes
  to zero.
  \begin{enumerate}
  \item
  \label{prob.MW.1}
    What is the enclosed mass at each radius, $M(<r)$?
  \item
  \label{prob.MW.2}
    If a star or a blob of gas of mass $m$ is in a circular orbit
    around the Milky Way (with $r < R$), what is its velocity $v(r)$?
    How does this scale with radius?\footnote{If a function ``scales
      with radius'' as $r^\alpha$, this is commonly denoted as $f(r)
      \propto r^\alpha$.}
  \item
    \label{prob.MW.3}
    What about at $r > R$? How does $v(r)$ scale with radius outside
    the galaxy?
  \item
    \label{prob.MW.4}
    Sketch the entire profile $v(r)$ as a function of
    radius, from $r=0$ to $r \gg R$. (\textit{HINT: this is most
      easily shown as a ``log-log'' plot, which Ben will discuss}).
  \item
    \label{prob.MW.5}
    Even though the vast majority of the visible mass of the Milky Way
    (stars, gas, etc.) is contained within about 15 kpc, there are a
    few objects orbiting far outside the central galaxy, allowing us
    to measure $v(r)$. Instead of decreasing with radius, $v(r)$
    appears to remain constant out to several hundred kpc. What do you
    think this means? 
  \end{enumerate}

\end{enumerate}

\end{document}
